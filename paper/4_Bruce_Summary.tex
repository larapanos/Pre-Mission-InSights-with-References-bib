This paper summarizes the state of knowledge of the interior structure and composition of Mars.  Our understanding of Mars’ evolution is highly dependent on estimates of key parameters.  InSight will determine Mars’ interior layering, core state, heat flux, and level of tectonic activity and meteorite impact rates Table~\ref{tab:uncertainty}. Uncertainty requirements are specified to levels necessary to both enable comparative planetology studies and to vastly improve the input parameters for models of thermal and chemical evolution of the interior.  At present, many of these parameters are very loosely constrained via indirect measurements Table~\ref{tab:uncertainty}. InSight will vastly decrease uncertainties, in some cases to ranges approaching those estimated for Earth.  Since the InSight proposal was written in 2010, only the estimate of core size has substantially improved.  As discussed above, new data for the planetary gravity field and rover tracking data have provided better k2 constraints, decreasing the uncertainty in core radius to approximately ± 100 km.  Core density is still poorly constrained and will be significantly aided by InSight measurements of core size and state.

Each individual measurement is an important key to Mars’ past evolution and current state. Collectively these data will yield a tremendous leap forward for understanding the tightly coupled processes of planetary thermal and chemical evolution.  For example, both seismic measurements and heat flow constrain present day temperature.   The mantle temperature is function of the concentration of radiogenic elements, any phase transitions, and heat coming from the core.  Seismic velocity helps constrain mantle temperature and water content.  The heat flux measured at the surface is the sum of mantle and core heat flux, with the dominant contribution from radiogenic elements in the crust.  By measuring crustal thickness, it becomes possible to constrain the relative contributions of the crustal and mantle contributions.  Similarly, these parameters are all interconnected with respect to Mars’ bulk composition.

The availability of meteorites and datasets from dozens of missions make Mars the best studied rocky planet beyond our own.  The data from InSight are necessary to make a huge leap forward in understanding Martian history, from its earliest formation, to differentiation, to formation of its atmosphere.  Many fundamental questions remain for Mars, such as: Is the interior composition chondritic? What phase transitions exist in the mantle?  Is the core fully liquid? Does the crust have layers? Are layers preserved from a magma ocean? What kinds of geologically processes are active today?  At what depth should water be liquid today, if present?  

Knowledge of rocky planetary interior structure and evolution is primarily based upon the abundant datasets available for the Earth, and secondarily from much more limited data for the Moon.  Our understanding of magma ocean processes and planetary differentiation is dominated by these two bodies with very different sizes and geologic histories.   The InSight data set will provide a window into a rocky planet that is large enough to have enough heat production to have an extended geologic history, but arguably small enough such that heat loss from the interior does not appear to have driven plate tectonics.  Unlike the Moon, Mars has an atmosphere and hydrosphere (now a cryosphere) a result of its larger size and extended volcanic activity.  Unlike Earth, stable surface water on Mars was short-lived.  In these and many other regards, Mars is substantially different from Earth and the Moon.  Data from InSight will enable a better understanding of the continuum of rocky bodies.




\begin{table}[h!]
\centering
\caption{Current state of knowledge of the interior of Mars and expected levels of uncertainty at the end of the InSight mission.}
\begin{tabular}{|l|l|l|}
\hline
\multicolumn{1}{|c|}{\textbf{Quantity}} & \multicolumn{1}{c|}{\textbf{Current Uncertainty}} & \multicolumn{1}{c|}{\textbf{Science Requirement}} \\ \hline
Crust Thickness                & ±35 km (inferred)                        & ±10 km                                   \\ \hline
Crust Layering                 & No knowledge                             & 0.5 km/s contrasts                       \\ \hline
Mantle Seismic Velocity        & ±1 km/s (inferred)                       & ±0.25 km/s                               \\ \hline
Core Radius                    & ±450 km                                  & ±200 km                                  \\ \hline
Core Density                   & ±1000 kg/m3                              & ±450 kg/m3                               \\ \hline
Core State                     & Likely liquid                            & Positive det.                            \\ \hline
Heat Flux                      & ±25 mW/m2 (inferred)                     & ±5 mW/m2                                 \\ \hline
Level of Activity              & Factor of 10 (inferred)                  & Factor of 2                              \\ \hline
Location of Activity           & No knowledge                             & ±25\% of distance                        \\ \hline
Impact Rate                    & Factor of 6                              & Factor of 2                              \\ \hline
\end{tabular}
\label{tab:uncertainty}
\end{table}


